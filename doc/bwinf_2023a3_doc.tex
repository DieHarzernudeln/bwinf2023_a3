\makeatletter
\def\input@path{{libs/}} % Add the subdirectory to the search path
\makeatother

\documentclass[a4paper,10pt,ngerman]{scrartcl}
\usepackage{babel}
\usepackage[T1]{fontenc}
\usepackage[utf8x]{inputenc}
\usepackage{listings-rust}
\usepackage[a4paper,margin=2.5cm,footskip=0.5cm]{geometry}

% Die nächsten vier Felder bitte anpassen:
\newcommand{\Aufgabe}{Aufgabe 3: Zauberschule}  % Aufgabennummer und Aufgabennamen angeben
\newcommand{\TeamId}{00054}                     % Team-ID aus dem PMS angeben
\newcommand{\TeamName}{K7Bots}                  % Team-Namen angeben
\newcommand{\Namen}{Felix Mölle}                % Namen der Bearbeiter/-innen dieser Aufgabe angeben
 
% Kopf- und Fußzeilen
\usepackage{scrlayer-scrpage, lastpage}
\setkomafont{pageheadfoot}{\large\textrm}
\lohead{\Aufgabe}
\rohead{Team-ID: \TeamId}
\cfoot*{\thepage{}/\pageref{LastPage}}

% Position des Titels
\usepackage{titling}
\setlength{\droptitle}{-1.0cm}
\definecolor{mygreen}{rgb}{0,0.6,0}
\definecolor{mygray}{rgb}{0.5,0.5,0.5}
\definecolor{mymauve}{rgb}{0.58,0,0.82}
\lstset{
  keywordstyle=\color{blue},commentstyle=\color{mygreen},
  stringstyle=\color{mymauve},rulecolor=\color{black},
  basicstyle=\footnotesize\ttfamily,numberstyle=\tiny\color{mygray},
  captionpos=b, % sets the caption-position to bottom
  keepspaces=true, % keeps spaces in text
  numbers=left, numbersep=5pt, showspaces=false,showstringspaces=true,
  showtabs=false, stepnumber=2, tabsize=2, title=\lstname
  }
\lstdefinelanguage{JavaScript}{ % JavaScript ist als einzige Sprache noch nicht vordefiniert
  keywords={break, case, catch, continue, debugger, default, delete, do, else, finally, for, function, if, in, instanceof, new, return, switch, this, throw, try, typeof, var, void, while, with},
  morecomment=[l]{//},
  morecomment=[s]{/*}{*/},
  morestring=[b]',
  morestring=[b]",
  sensitive=true
}

% Diese beiden Pakete müssen zuletzt geladen werden
%\usepackage{hyperref} % Anklickbare Links im Dokument
\usepackage{cleveref}

% Daten für die Titelseite
\title{\textbf{\Huge\Aufgabe}}
\author{\LARGE Team-ID: \LARGE \TeamId \\\\
	    \LARGE Team-Name: \LARGE \TeamName \\\\
	    \LARGE Bearbeiter/-innen dieser Aufgabe: \\ 
	    \LARGE \Namen\\\\}
\date{\LARGE\today}

\begin{document}

\maketitle
\tableofcontents

\vspace{0.5cm}

\section{Lösungsidee}
\begin{flushleft}
  Die Lösungsidee besteht darin, den A*-Algorithmus zu verwenden, um den optimalen Pfad von A nach B zu finden.\\
  Der A* Algorithmus ist geeignet da er sehr effizient ist. \\
  Durch die Verwendung der \glqq{Luftlinie}\grqq{} zum Ziel ist er 
  z.B. dem Dijkstra-Algorithmus überlegen\\
  Da z.T. sehr große Ebenen verwendet werden, ist auch das Überprüfen aller möglicher Wege nicht sinnvoll
  \linebreak
  \linebreak
  Dazu wird die gegebene Textdatei zuerst eingelesen und in eine verwendbare Datenstruktur 
  \\
  (z.B. 3 Dimensionales Array) überführt
  \linebreak
  \linebreak
  Dabei wird ein Suchbaum erstellt, wobei jeder Knoten einen Zustand repräsentiert, der aus Rons aktueller Position, dem Abstand zum Startpunkt (g), dem kürzest möglichen Abstand zum Ziel (h) und dem Vorgängerknoten besteht.
  \\
  Der Algorithmus besitzt 2 Listen. Die \glqq{openList}\grqq{} enthält alle zu untersuchenden Knoten und die \glqq{closedList}\grqq{} alle bereits abschließend untersuchten Knoten.
  \\
  Der Algorithmus bewertet die Knoten basierend auf den Kosten für das Erreichen des aktuellen Knotens und einer geschätzten Kostenfunktion für das Erreichen des Ziels.
  \linebreak
  \linebreak
  Folgendes geschieht solange es noch zu untersuchende Knoten in der \glqq{openList}\grqq{} gibt:
  \begin{enumerate}

    \item Es wird immer der Knoten (x) mit dem geringsten Wert von $f=g+h$ überprüft.
    \item Wenn (x) bereits das Ziel ist bricht der Algorithmus ab.
    \item (x) wird aus der \glqq{openList}\grqq{} entfernt und als abschließend untersuchter Knoten eingetragen
    \item Es werden für alle alle begehbaren Nachbarn (n) von (x) die g und h Werte berechnet
          \subitem{Dabei gilt $g_{n} = g_{x} + dist(x,n); dist(x,n) \in \{1;3\}$ und $h_{n} = optimal\_dist(n,target)$}
    \item Die Nachbarn werden mitsamt ihrer g und h Werte in die \glqq{openList}\grqq{} eingetragen.

  \end{enumerate}
  Durch die Expansion der Knoten und die Auswahl desjenigen mit der geringsten Gesamtkostenbewertung wird der kürzeste Weg ermittelt.
  Um die Wände zu berücksichtigen, wird bei der Erstellung des Nachbarknotens geprüft, ob dieser begehbar ist.\\
  Im Generellen wird immer der Übergang in das jeweils andere Stockwerk in Betracht gezogen,
  wobei diese Wege durch den hohen g Wert tendentiell unattraktiv sind.\\
\end{flushleft}

\section{Umsetzung}
Die Umsetzung dieser Aufgabe erfolgte in Rust, einer Programmiersprache, die für ihre Sicherheit, Geschwindigkeit und Nebenläufigkeit bekannt ist.
Der Code ist in folgende Module aufgeteilt, um die Lesbarkeit und Wartbarkeit zu verbessern:
\begin{itemize}
    \item main.rs ->
          Dieses Modul enthält die main-Funktion, die das Programm startet und die Eingabedaten liest.
          Es ruft den A*-Algorithmus auf, um den kürzesten Weg zu finden, und gibt die Ergebnisse aus.

    \item utils.rs ->
          In diesem Modul werden Hilfsfunktionen und Datenstrukturen definiert, die in verschiedenen Teilen des Programms verwendet werden.
          Dazu gehören das Einlesen der Eingabedaten, die Konvertierung von Zeichen in Zahlen und Datenstrukturen wie PVector und Node.

    \item astar.rs ->
          Hier wird der A*-Algorithmus implementiert, um den optimalen Pfad zu finden.
          Dieses Modul enthält die Hauptlogik zur Pfadberechnung.
\end{itemize}

\subsection{Datenstrukturen}
\subsubsection*{PVector}
Die PVector-Struktur repräsentiert einen 3D-Vektor, der in diesem Kontext Rons Position auf dem Spielfeld darstellt.\\
Sie speichert die Koordinaten für x, y und das Stockwerk z.
Diese Struktur dient als Key in der Nodes Hashmap um eine Node im A*-Algorithmus zu identifizieren.
\subsubsection*{Node}
Die Node-Struktur repräsentiert einen Knoten im Suchbaum des A*-Algorithmus. \\
Sie enthält Informationen über die Kosten g (bisherige Kosten, um diesen Knoten zu erreichen),
die geschätzten Kosten h (geschätzte Kosten, um vom aktuellen Knoten zum Ziel zu gelangen)
und einen Verweis auf den Vorgängerknoten (Parent).

\subsection{Einlesen der Eingabedaten}
Die Eingabedaten werden in \texttt{utils.rs} verarbeitet.
Die Funktion \texttt{read\_data()} liest die Dimensionen des Spielfelds, die Positionen von Ron und dem Ziel,
sowie die Spielfeldkonfiguration (Wände, Felder usw.) aus einer Datei oder von der Benutzereingabe.

\subsection{Konvertierung von Zeichen in Zahlen}
Die Funktion \texttt{char\_to\_int()} in \texttt{utils.rs} konvertiert Zeichen aus der Spielfeldkonfiguration in entsprechende Zahlenwerte.
Dies ermöglicht die einfache Identifikation von Wänden, Feldern, Rons Position und dem Ziel auf dem Spielfeld.

Diese Beschreibung gibt einen umfassenden Überblick über die Umsetzung der Aufgabe. Der Code verwendet Rusts Leistungsfähigkeit und Sicherheit, um den kürzesten Weg unter Berücksichtigung der gegebenen Bedingungen in der Zauberschule Bugwarts zu finden.

\subsection{A* Algorithmus}
Der A*-Algorithmus wird in astar.rs implementiert und ist der Kern des Programms.

Der Algorithmus arbeitet wie folgt:

\begin{itemize}

    \item Initialisierung der offenen Liste (known\_nodes) und der leeren geschlossenen Liste (finished\_nodes) als Vec.

    \item Initialisierung der Map für die Nodes als HashMap<PVector, Node>

    \item Startknoten wird zur offenen Liste hinzugefügt. Dieser Knoten enthält Informationen über die aktuelle Position von Ron.

    \item Während die offene Liste nicht leer ist, wird der Knoten mit den geringsten Gesamtkosten (g + h) ausgewählt.\\
          \texttt{let current = *known\_nodes.iter().min\_by\_key(|p| nodes[p].g + nodes[p].h).unwrap();}

    \item Wenn der ausgewählte Knoten das Ziel ist, wird der Pfad über die parents zurückverfolgt\\ und die Kosten sowie der gefundene Pfad werden ausgegeben.

    \item Andernfalls werden die Nachbarknoten des ausgewählten Knotens generiert und deren Kosten berechnet. Wenn ein Nachbarnode bereits in der geschlossenen Liste ist und die Kosten geringer sind, wird er übersprungen. Andernfalls wird er zur offenen Liste hinzugefügt.

    \item Schritte 4 bis 6 werden wiederholt, bis das Ziel erreicht ist oder festgestellt wird, dass kein Pfad existiert.

\end{itemize}

\section{Beispiele}
\subsection{Zauberschule 0}

\begin{verbatim}
cost: 8; path: [
PVector { x: 5, y: 9, z: 0 },
PVector { x: 5, y: 9, z: 1 },
PVector { x: 6, y: 9, z: 1 },
PVector { x: 7, y: 9, z: 1 },
PVector { x: 7, y: 9, z: 0 }]
\end{verbatim}

\subsection{Zauberschule 1}

\begin{verbatim}
cost: 4; path: [
PVector { x: 13, y: 7, z: 0 },
PVector { x: 12, y: 7, z: 0 },
PVector { x: 11, y: 7, z: 0 },
PVector { x: 11, y: 6, z: 0 },
PVector { x: 11, y: 5, z: 0 }]
\end{verbatim}

\subsection{Zauberschule 2}

\begin{verbatim}
cost: 14; path: [
PVector { x: 21, y: 3, z: 0 },
PVector { x: 22, y: 3, z: 0 },
...
PVector { x: 26, y: 5, z: 0 },
PVector { x: 27, y: 5, z: 0 }]
\end{verbatim}

\subsection{Zauberschule 3}

\begin{verbatim}
cost: 28; path: [
PVector { x: 3, y: 27, z: 0 },
...
PVector { x: 21, y: 25, z: 0 }]
\end{verbatim}

\subsection{Zauberschule 4}

\begin{verbatim}
cost: 84; path: [
PVector { x: 73, y: 67, z: 0 },
...
PVector { x: 31, y: 55, z: 0 }]
\end{verbatim}

\subsection{Zauberschule 5}

\begin{verbatim}
cost: 124; path: [
PVector { x: 13, y: 167, z: 0 },
PVector { x: 13, y: 167, z: 1 },
...
PVector { x: 72, y: 155, z: 0 },
PVector { x: 71, y: 155, z: 0 }]
\end{verbatim}

\subsection{My\_Zauberschule 6}

\begin{verbatim}
cost: 364; path: [
PVector { x: 1, y: 1, z: 0 },
PVector { x: 2, y: 1, z: 0 },
...
PVector { x: 79, y: 198, z: 0 },
PVector { x: 79, y: 199, z: 0 }]
\end{verbatim}

\section{Quellcode}
\subsection{main.rs: -> Hauptprogramm}

\begin{lstlisting}[language=Rust, style=colouredRust]
  use crate::utils::read_data;
  use std::env;

  mod astar;
  mod utils;

  fn main() {
      let args: Vec<String> = env::args().collect();
      let mut fixed_path = String::new();

      if args.len() >= 2 {
          fixed_path = args[1].clone();
      }
      let (size_x, size_y, floors, start, target) = read_data(fixed_path);
      //println!("start: {:?}; target: {:?}", start, target);

      let (cost, path) = astar::astar(floors, start, target, size_x, size_y);
      println!("cost: {:?}; path: {:?}", cost, path)
  }
\end{lstlisting}

\subsection{ astar.rs -> Die eigentliche Implementierung des a*-Algorithmus:}
\begin{lstlisting}[language=Rust, style=colouredRust]
use crate::utils::{Node, PVector};
use std::collections::HashMap;

pub fn astar(
  floors: [Vec<Vec<u8>>; 2], start: PVector, target: PVector, size_x: u32, size_y: u32
  ) -> (u32, Vec<PVector>) {
    let mut nodes: HashMap<PVector, Node> = HashMap::new();

    let mut known_nodes: Vec<PVector> = vec![];
    let mut finished_nodes: Vec<PVector> = vec![];

    known_nodes.push(start);
    nodes.insert(start, Node::new(0, 0, PVector::zero()));

    while !known_nodes.is_empty() {
        let current = *known_nodes
            .iter()
            .min_by_key(|p| nodes[p].g + nodes[p].h)
            .unwrap();

        if current == target {
            return finalize(start, target, &nodes);
        }

        known_nodes.retain(|&p| p != current);
        finished_nodes.push(current);

        let neighbors = get_neighbors(current, &floors, size_x, size_y);
        check_neighbors(&neighbors, current, &mut nodes, &mut known_nodes, &finished_nodes, target);
    }

    (0, vec![])
}

fn check_neighbors(
    neighbors: &HashMap<PVector, u32>, current: PVector,
    nodes: &mut HashMap<PVector, Node>, known_nodes: &mut Vec<PVector>,
    finished_nodes: &[PVector], target: PVector,
) {
    for neighbor in neighbors.keys() {
        let dist = *neighbors.get(neighbor).unwrap();

        if finished_nodes.contains(neighbor) && nodes[neighbor].g <= nodes[&current].g + dist {
            continue;
        }

        let g = nodes[&current].g + dist;
        let h = neighbor.dist(target);

        if !known_nodes.contains(neighbor) {
            known_nodes.push(*neighbor);
        } else if g >= nodes[&neighbor].g {
            continue;
        }

        nodes.insert(*neighbor, Node::new(g, h, current));
    }
}

fn get_neighbors(current: PVector, floors: &[Vec<Vec<u8>>; 2], size_x: u32, size_y: u32)
 -> HashMap<PVector, u32> {
    let mut neighbors: HashMap<PVector, u32> = HashMap::new();

    if current.x > 0 {
      neighbors.insert(PVector::new(current.x - 1, current.y, current.z), 1);}
    if current.x < size_x - 1 {
      neighbors.insert(PVector::new(current.x + 1, current.y, current.z), 1);}
    if current.y > 0 {
      neighbors.insert(PVector::new(current.x, current.y - 1, current.z), 1);}
    if current.y < size_y - 1 {
      neighbors.insert(PVector::new(current.x, current.y + 1, current.z), 1);}
    if current.z == 1 {
      neighbors.insert(PVector::new(current.x, current.y, current.z - 1), 3);}
    if current.z == 0 {
      neighbors.insert(PVector::new(current.x, current.y, current.z + 1), 3);}

    return neighbors
        .iter()
        .filter(|n| is_walkable(n.0, floors))
        .map(|(k, v)| (*k, *v))
        .collect();
}

fn is_walkable(current: &PVector, floors: &[Vec<Vec<u8>>; 2]) -> bool {
    floors[current.z as usize][current.y as usize][current.x as usize] != 0
}


fn finalize(
    start: PVector, target: PVector, nodes: &HashMap<PVector, Node>,
) -> (u32, Vec<PVector>) {
    let mut path: Vec<PVector> = vec![];
    let mut current = target;
    while current != start {
        path.push(current);
        current = nodes[&current].parent;
    }
    path.push(start);
    path.reverse();
    (nodes[&target].g, path)
}

\end{lstlisting}

\subsection{utils.rs -> FileReading und struct Definitionen}
\begin{lstlisting}[language=Rust, style=colouredRust]
use std::collections::HashMap;
use std::io::{BufRead, BufReader};

#[derive(Debug, Clone, Copy, Hash, Eq, PartialEq)]
pub struct PVector {
    pub x: u32,
    pub y: u32,
    pub z: u32,
}

impl PVector {
    pub fn new(x: u32, y: u32, z: u32) -> Self {
        PVector { x, y, z }
    }

    pub fn zero() -> Self {
        PVector { x: 0, y: 0, z: 0 }
    }

    pub fn dist(&self, pos: PVector) -> u32 {
        self.x.abs_diff(pos.x) + self.y.abs_diff(pos.y) + self.z.abs_diff(pos.z) * 3
    }
}

#[derive(Debug)]
pub struct Node {
    pub g: u32,
    pub h: u32,
    pub parent: PVector,
}

impl Node {
    pub fn new(g: u32, h: u32, parent: PVector) -> Self {
        Node { g, h, parent }
    }
}

pub fn read_data(fixed_path: String) -> (u32, u32, [Vec<Vec<u8>>; 2], PVector, PVector) {

    //Stark gekuerzt
    let mut input = String::new();
    //Entweder fixed_path oder Nutzereingabe als Dateipfad verwenden
    let mut lines = //Datei lesen und Zeilen speichern

    let size_y = //Erste Zahl in erster Zeile
    let size_x = //Zweite Zahl in erster Zeile

    let mut start: PVector = PVector::zero();
    let mut target: PVector = PVector::zero();

    let mut floors: [Vec<Vec<u8>>; 2] = [Vec::new(), Vec::new()];

    //Zeilen einlesen und in das Vec-Array speichern

    (size_x, size_y, floors, start, target)
}

fn char_to_int(ch: char) -> u8 {
  //mapping von '#' auf 0; '.' auf 1; 'A' auf 2 und 'B' auf 3
  //da arbeiten auf int einfacher als auch char
}

\end{lstlisting}


\end{document}
