\section{Quellcode}
\subsection{main.rs: -> Hauptprogramm}

\begin{lstlisting}[language=Rust, style=colouredRust]
  use crate::utils::read_data;
  use std::env;

  mod astar;
  mod utils;

  fn main() {
      let args: Vec<String> = env::args().collect();
      let mut fixed_path = String::new();

      if args.len() >= 2 {
          fixed_path = args[1].clone();
      }
      let (size_x, size_y, floors, start, target) = read_data(fixed_path);
      //println!("start: {:?}; target: {:?}", start, target);

      let (cost, path) = astar::astar(floors, start, target, size_x, size_y);
      println!("cost: {:?}; path: {:?}", cost, path)
  }
\end{lstlisting}

\subsection{ astar.rs -> Die eigentliche Implementierung des a*-Algorithmus:}
\begin{lstlisting}[language=Rust, style=colouredRust]
use crate::utils::{Node, PVector};
use std::collections::HashMap;

pub fn astar(
  floors: [Vec<Vec<u8>>; 2], start: PVector, target: PVector, size_x: u32, size_y: u32
  ) -> (u32, Vec<PVector>) {
    let mut nodes: HashMap<PVector, Node> = HashMap::new();

    let mut known_nodes: Vec<PVector> = vec![];
    let mut finished_nodes: Vec<PVector> = vec![];

    known_nodes.push(start);
    nodes.insert(start, Node::new(0, 0, PVector::zero()));

    while !known_nodes.is_empty() {
        let current = *known_nodes
            .iter()
            .min_by_key(|p| nodes[p].g + nodes[p].h)
            .unwrap();

        if current == target {
            return finalize(start, target, &nodes);
        }

        known_nodes.retain(|&p| p != current);
        finished_nodes.push(current);

        let neighbors = get_neighbors(current, &floors, size_x, size_y);
        check_neighbors(&neighbors, current, &mut nodes, &mut known_nodes, &finished_nodes, target);
    }

    (0, vec![])
}

fn check_neighbors(
    neighbors: &HashMap<PVector, u32>, current: PVector,
    nodes: &mut HashMap<PVector, Node>, known_nodes: &mut Vec<PVector>,
    finished_nodes: &[PVector], target: PVector,
) {
    for neighbor in neighbors.keys() {
        let dist = *neighbors.get(neighbor).unwrap();

        if finished_nodes.contains(neighbor) && nodes[neighbor].g <= nodes[&current].g + dist {
            continue;
        }

        let g = nodes[&current].g + dist;
        let h = neighbor.dist(target);

        if !known_nodes.contains(neighbor) {
            known_nodes.push(*neighbor);
        } else if g >= nodes[&neighbor].g {
            continue;
        }

        nodes.insert(*neighbor, Node::new(g, h, current));
    }
}

fn get_neighbors(current: PVector, floors: &[Vec<Vec<u8>>; 2], size_x: u32, size_y: u32)
 -> HashMap<PVector, u32> {
    let mut neighbors: HashMap<PVector, u32> = HashMap::new();

    if current.x > 0 {
      neighbors.insert(PVector::new(current.x - 1, current.y, current.z), 1);}
    if current.x < size_x - 1 {
      neighbors.insert(PVector::new(current.x + 1, current.y, current.z), 1);}
    if current.y > 0 {
      neighbors.insert(PVector::new(current.x, current.y - 1, current.z), 1);}
    if current.y < size_y - 1 {
      neighbors.insert(PVector::new(current.x, current.y + 1, current.z), 1);}
    if current.z == 1 {
      neighbors.insert(PVector::new(current.x, current.y, current.z - 1), 3);}
    if current.z == 0 {
      neighbors.insert(PVector::new(current.x, current.y, current.z + 1), 3);}

    return neighbors
        .iter()
        .filter(|n| is_walkable(n.0, floors))
        .map(|(k, v)| (*k, *v))
        .collect();
}

fn is_walkable(current: &PVector, floors: &[Vec<Vec<u8>>; 2]) -> bool {
    floors[current.z as usize][current.y as usize][current.x as usize] != 0
}


fn finalize(
    start: PVector, target: PVector, nodes: &HashMap<PVector, Node>,
) -> (u32, Vec<PVector>) {
    let mut path: Vec<PVector> = vec![];
    let mut current = target;
    while current != start {
        path.push(current);
        current = nodes[&current].parent;
    }
    path.push(start);
    path.reverse();
    (nodes[&target].g, path)
}

\end{lstlisting}

\subsection{utils.rs -> FileReading und struct Definitionen}
\begin{lstlisting}[language=Rust, style=colouredRust]
use std::collections::HashMap;
use std::io::{BufRead, BufReader};

#[derive(Debug, Clone, Copy, Hash, Eq, PartialEq)]
pub struct PVector {
    pub x: u32,
    pub y: u32,
    pub z: u32,
}

impl PVector {
    pub fn new(x: u32, y: u32, z: u32) -> Self {
        PVector { x, y, z }
    }

    pub fn zero() -> Self {
        PVector { x: 0, y: 0, z: 0 }
    }

    pub fn dist(&self, pos: PVector) -> u32 {
        self.x.abs_diff(pos.x) + self.y.abs_diff(pos.y) + self.z.abs_diff(pos.z) * 3
    }
}

#[derive(Debug)]
pub struct Node {
    pub g: u32,
    pub h: u32,
    pub parent: PVector,
}

impl Node {
    pub fn new(g: u32, h: u32, parent: PVector) -> Self {
        Node { g, h, parent }
    }
}

pub fn read_data(fixed_path: String) -> (u32, u32, [Vec<Vec<u8>>; 2], PVector, PVector) {

    //Stark gekuerzt
    let mut input = String::new();
    //Entweder fixed_path oder Nutzereingabe als Dateipfad verwenden
    let mut lines = //Datei lesen und Zeilen speichern

    let size_y = //Erste Zahl in erster Zeile
    let size_x = //Zweite Zahl in erster Zeile

    let mut start: PVector = PVector::zero();
    let mut target: PVector = PVector::zero();

    let mut floors: [Vec<Vec<u8>>; 2] = [Vec::new(), Vec::new()];

    //Zeilen einlesen und in das Vec-Array speichern

    (size_x, size_y, floors, start, target)
}

fn char_to_int(ch: char) -> u8 {
  //mapping von '#' auf 0; '.' auf 1; 'A' auf 2 und 'B' auf 3
  //da arbeiten auf int einfacher als auch char
}

\end{lstlisting}
